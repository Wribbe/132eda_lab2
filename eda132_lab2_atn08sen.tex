\documentclass{article}

\usepackage[utf8]{inputenc}
\usepackage[T1]{fontenc}
\usepackage[english]{babel}
\usepackage{amsmath}

\setlength{\parindent}{0cm}

\begin{document}
  \begin{center}
    EDA132 -- Assignment 3 (prev. Assignment 2)\\
    \ \\
    {\Large Robot localisation with HMM based forward-filtering.} \\
    \ \\
    Stefan Eng <\texttt{atn08sen@student.lu.se}> \\
    \ \\
    ---
  \end{center}
  \vspace{-0.8cm}
  \section*{Introduction}

    The following assignment task is to locate, as precisely as possible, the
    discrete x- and y-coordinates of a hidden robot in a (X,Y)-sized
    grid-world, using the robots faulty sensor.

  \section*{Method}

    It is known that the robot orients itself according to the following
    probabilities:
    \begin{align*}
      P(\textrm{Keep orientation | Facing wall}) &= 0.0 \\
      P(\textrm{Keep orientation | Not facing wall}) &= 0.7 \\
      P(\textrm{Keep orientation | Not facing wall}) &= 0.3
    \end{align*}

    There is also a known distribution of the readings coming from the faulty
    sensor:
    \begin{align*}
      P(\textrm{True position}) &= 0.1 \\
      P(\textrm{1-step-off}) &= 0.05 \cdot np1\\
      P(\textrm{2-step-off}) &= 0.025 \cdot np2\\
      P(\textrm{Nothing}) &= 1.0-\Sigma P(\textrm{\{True,1-,2-off\}})
    \end{align*}

    Where $np1$ and $np2$ are the number of possible (inside the grid)
    positions one and two steps from the robots true location respectively.
    Additionally the assumptions that the robot always successfully moves one
    tile and stays inside the grid are made. \\

    \textbf{Answer Q1:} The sensor model could be used in order to stay away
    from the walls by assuming that increased nothing-readings indicate that
    the robot is approaching the walls and corners. \\

    \textbf{Encoding:} Since there are four possible directions; N, E, S, W
    and X$\cdot$Y number of tiles there are X${\cdot}$Y${\cdot}4$ possible
    states on the grid. These states will be represented as a list of numbers
    in Python according to the following scheme.

    \begin{center}
      \texttt{grid = [0 1 2 3 4 5 6 7 ...\ X$\cdot$Y$\cdot$4-1] }
    \end{center}

    Where the digits at position 0,1,2 and 3 represent the values of
    \texttt{grid[0][0][N,E,W,S]}, and 4,5,6 and 7 the values of
    \texttt{grid[0][1][N,E,W,S]} for \texttt{grid[x][y][direction]}.


  \section*{Results}
  \section*{Discussion}
  \section*{Implementation}
\end{document}
